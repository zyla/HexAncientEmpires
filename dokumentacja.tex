\documentclass[paper=a4, fontsize=11pt]{scrartcl} % A4 paper and 11pt font size

\usepackage[T1]{fontenc} % Use 8-bit encoding that has 256 glyphs
\usepackage{fourier} % Use the Adobe Utopia font for the document - comment this line to return to the LaTeX default
\usepackage{polski} %język polski
\usepackage[utf8]{inputenc}
\usepackage{amsmath,amsfonts,amsthm} % Math packages
\usepackage{listings}
\lstset{language=C++} 

\usepackage{lipsum} % Used for inserting dummy 'Lorem ipsum' text into the template

\usepackage{sectsty} % Allows customizing section commands
\allsectionsfont{\centering \normalfont\scshape} % Make all sections centered, the default font and small caps

\usepackage{fancyhdr} % Custom headers and footers
\pagestyle{fancyplain} % Makes all pages in the document conform to the custom headers and footers
\fancyhead{} % No page header - if you want one, create it in the same way as the footers below
\fancyfoot[L]{} % Empty left footer
\fancyfoot[C]{} % Empty center footer
\fancyfoot[R]{\thepage} % Page numbering for right footer
\renewcommand{\headrulewidth}{0pt} % Remove header underlines
\renewcommand{\footrulewidth}{0pt} % Remove footer underlines
\setlength{\headheight}{13.6pt} % Customize the height of the header

\numberwithin{equation}{section} % Number equations within sections (i.e. 1.1, 1.2, 2.1, 2.2 instead of 1, 2, 3, 4)
\numberwithin{figure}{section} % Number figures within sections (i.e. 1.1, 1.2, 2.1, 2.2 instead of 1, 2, 3, 4)
\numberwithin{table}{section} % Number tables within sections (i.e. 1.1, 1.2, 2.1, 2.2 instead of 1, 2, 3, 4)

\setlength\parindent{0pt} % Removes all indentation from paragraphs - comment this line for an assignment with lots of text

%----------------------------------------------------------------------------------------
%	TITLE SECTION
%----------------------------------------------------------------------------------------

\newcommand{\horrule}[1]{\rule{\linewidth}{#1}} % Create horizontal rule command with 1 argument of height

\title{	
\normalfont \normalsize 
\textsc{Politechnika Warszawska} \\ [25pt] % Your university, school and/or department name(s)
\huge HexAncientEmpires \\ % The assignment title
}

\author{Tomasz Zieliński,  Maciej Bielecki}

\date{\normalsize\today} 

\begin{document}
\maketitle 
%------------------------------------------------

\section{Wstęp}
Nasz projekt to odtworzenie częci aspektów gry Ancient Empires II wraz z naszymi modyfikacjami. Najpoważniejszą zmianą było zastąpienie planszy klasycznej przez mape podzieloną na szeciokąty. Ograniczylimy trochę zawiłe szczegółyrozgrywki, by gra była bardziej przystepna dla użytkownika. Gra daziała pod systamem Android i wykorzystuje komunikację Bluetooth. 
%------------------------------------------------

\section{Opis Gry}
Rozgrywkę rozpoczyna gracz pierwszy. Do dyspozycji ma jednostki w kolorze odpowiadającym numerowi gracza. Są one rozmieszczone na mapie.
Gracz może w swojej turze jednokrotnie poruszyć i jednokrotnie zaatakować każdą ze swoich jednostek. Jednostki poruszają się po mapie w nietrywialny sposób, ponieważ mogą znajdować się tylko na siatce, a koszt wejcia na konkretny typ podłoża jest rózny. Żeby poruszyć jednostką należy umiecić na niej kursor, a następnie wybrać pozycje z tych oznaczonych przez kolor niebieski. Tylko takie miejsca są w zasięgu poruszania danej jednostki. Jednostka następnie wykona płynny ruch reprezentowany animacją i znajdze się w wyznaczonym celu. Od tego momentu nie może poruszyc się aż do końca tury.  Wartoć ataku, zasięg poruszanie i zasięg ataku jednostek zależy od ich typu. Jednostka będąc atakowana otrzymuje obrażenia zmniejszone o obrone pola, na którym stoi
%------------------------------------------------

\section{Architektura}
connect acticity i game actictiwity  i to co obsłóguje połączenia ConnectionService
Program wykonyje dwie główne aktywnoci reprezentowane prze ConnectActivity i GameActivity. ConnectActivity jest odpowiedzialne za nawiązanie połącznia BlueTooth. GameActivity jest odpowiedzialne za wykonywanie pozostałych zadań gry.
%------------------------------------------------

\section{Superblok}
W celu kontrolowania poprawności struktury logicznej powołany będzie superblok. Zawarte w nim zostaną 8B przechowujące odpowiednie informacje zależnie od wykonywaniej operacji:\\
- 1B - kod wykonywanej operacji\\
- 2B - adres początku pliku (numer bloku)\\
- 1B - sten aktualnie wykonywanej operacji\\
- 1B - warunek ukończenia wykonywanej operacji\\
- 1B - rozmiar pliku\\
- 1B – adres deskryptora (numer bloku)\\
- 2B - nazwa pliku\\
%------------------------------------------------

\section{Mapa Bitowa}
Realizowana jest na 8 B, z których każdy bit odwzorowuje jeden blok, czyli 2 B pamięci na dysku.
%------------------------------------------------

\section{Deskryptory plików}
Zawierają informacje o plikach. Każdy z nich jest zapisywany według następującego schematu:\\
- 2B – adres początku pliku (numer boku)\\
- 1B – długość pliku (w bajtach)\\ 
- 2B – nazwa pliku\\
%------------------------------------------------

\section{Plan Systamu Plików}
- 8 B – Superblok\\
- 80 B – tablica deskryptorów ( 5B * 16 )\\
- 8 B – mapa zajętości pamięci ( (1024 / 16) / 8 )\\
- 928 B – dane ( 1024B - 8B- 80B - 8B)\\
%------------------------------------------------

\section{Operacje W Systemie Plików}
- montowanie \\
- odczyt pliku \\
- usuwanie pliku\\
- zapis pliku \\
- odmontowanie \\
%------------------------------------------------

\section{Realizacja Operacji}
Każda z operacji zwraca wartość typu BOOL; TRUE w przypadku powodzenia operacji i FALSE w przypadku wystąpienia jakiegoś błędu. 

\subsection{Montowanie Systemu Plików}
- 1B - kod Montowania Systemu Plików\\
- 2B - adres obecnie obsłógiwanego miejsca\\
- 1B - numer deskryptora obsługiwanego pliku\\
- 1B - rozmiar obsługiwanego pliku\\
- 1B - nie używany w montowaniu\\
- 1B - nie używany w montowaniu\\
- 2B - nazwa pliku\\
\begin{samepage}
\begin{lstlisting}
if (stanSB() != OK)   dzialaj(); 
else 
{
    switch(stanSB()) 
	{
		case: ODCZYT_PLIKU: 
            zmienStatusSB(OK);  
            break; 
        case:  USUWANIE_PLIKU : 
            usunPlik(nazwaPliku); 
            break; 
        case:  TWORZENIE_DESKRYPTORA : 
            modyfikujDeskryptor( 
                SBAdres(),  
                SBRozmiar(),  
                SBNazwa() ); 
            break; 
        case:  USUWANIE_DESKRYPTORA :  
            modyfikujDeskryptor( 
                szukajDeskryptora(SBNazwa()), 
                0,  
                  ); 
            modyfikujSuperblok(0, 0, 0, 0,  OK ); 
            break; 
        default: 
            modyfikujSuperblok(0, 0, 0, 0,  OK ); 
            break; 
	}
}
\end{lstlisting}
\end{samepage}

\subsection{Kopiowanie Pliku}
- 1B - kod Kopiowania\\
- 2B - adres początku miejsca w które kopiujemy (numer bloku)\\
- 1B - liczba przekopiowanych bloków\\
- 1B - liczba bloków do przekopiowania\\
- 1B - rozmiar pliku\\
- 1B – adres deskryptora (numer bloku)\\
- 2B - nazwa pliku\\
\begin{samepage}
\begin{lstlisting}
if (wolneMiejsce() == rozmiarPliku)  
	error(); 
else if (wolneCiagleMiejsce() < rozmiarPliku)
{   
    if (zgodaNaDefragmentacje == true) 
		defragmentuj();  
    if (wolneMiejsce < rozmiarPliku)  
		error(); 
} 
else if (liczbaWolnychDeskryptorow() == 0)  
	error(); 
else 
{
    modyfikujSuperblok( 
        adresWolnegoMiejsca(),  
        rozmiarPliku,  
        adresWolnegoDeskryptora(),  
        nazwaPliku,  
         TWORZENIE_PLIKU ); 
 
    wprowadzDane(adresWolnegoMiejsca(), dane); 
 
    modyfikujDeskryptor( 
        adresWolnegoMiejsca(),  
        rozmiarPliku,  
        nazwaPliku); 
 
    modyfikujSuperblok(0, 0, 0, 0,  OK ); 
} 
return true;    
\end{lstlisting}
\end{samepage}

\subsection{Usuwanie Pliku}
- 1B - kod Usuwania\\
- 2B - adres początku pliku usuwanego (numer bloku)\\
- 1B - liczba wyzerowanych bloków\\
- 1B - liczba bloków do wyzerowania\\
- 1B - rozmiar pliku\\
- 1B – adres deskryptora (numer bloku)\\
- 2B - nazwa pliku\\
\begin{samepage}
\begin{lstlisting}
if (nazwaPliku != NULL) 
{ 
    if (szukajDeskryptora(nazwaPliku) != NULL)
	{
        adresBloku = adresBlokuPlikuZDeskryptora(szukajDeskryptora(nazwaPliku)); 
        modyfikujSuperblok( 
            adresBloku, 
            0,  
            szukajDeskryptora(nazwaPliku), 
            nazwaPliku,  
             USUWANIE_PLIKU ); 
         
        for (int i = adresBloku; i < adresBloku + CEIL(rozmiarPliku/2); i++) 
		{
            zerujBlok(i); 
		} 
        modyfikujSuperblok( 
            adresBloku, 
            0,  
            szukajDeskryptora(nazwaPliku), 
            nazwaPliku,  
             USUWANIE_DESKRYPTORA ); 
 
        modyfikujDeskryptor( 
            szukajDeskryptora(nazwaPliku), 
            0,  
              ); 
 
        modyfikujSuperblok(0, 0, 0, 0,  OK ); 
    }
	else 
        error(); 
} 
else 
    error(); 
\end{lstlisting}
\end{samepage}
%------------------------------------------------

\section{Czytanie plikue}
Prosta funkcja, która niekoniecznie musi zmieniać stan superbloku. My jednak mimo to, zmienimy flagę na ODCZYTPLIKU. Przekazujemy zawartość pliku, po czym ustawiamy wartości superbloku na zera i flagę na OK.
%------------------------------------------------

\section{Wypisywanie Plików na Woluminie}
Ze względu na to, iż nie zmieniamy wywołując tą funkcję nic w systemie plików, flaga w superbloku pozostaje bez zmian.  
- Sprawdzamy, czy 0-1B (adres początku pliku) są różne od zera oraz 3-4B (nazwa pliku) są niezerowe. Jeśli tak, pomijamy deskryptor i przechodzimy do następnego, jeśli nie, wypisujemy informację o pliku na ekran.

%------------------------------------------------
\end{document}