\documentclass[paper=a4, fontsize=11pt]{scrartcl} % A4 paper and 11pt font size

\usepackage[T1]{fontenc} % Use 8-bit encoding that has 256 glyphs
\usepackage{fourier} % Use the Adobe Utopia font for the document - comment this line to return to the LaTeX default
\usepackage{polski} %język polski
\usepackage[utf8]{inputenc}
\usepackage{amsmath,amsfonts,amsthm} % Math packages
\usepackage{listings}
\lstset{language=C++} 

\usepackage{lipsum} % Used for inserting dummy 'Lorem ipsum' text into the template

\usepackage{sectsty} % Allows customizing section commands
\allsectionsfont{\centering \normalfont\scshape} % Make all sections centered, the default font and small caps

\usepackage{fancyhdr} % Custom headers and footers
\pagestyle{fancyplain} % Makes all pages in the document conform to the custom headers and footers
\fancyhead{} % No page header - if you want one, create it in the same way as the footers below
\fancyfoot[L]{} % Empty left footer
\fancyfoot[C]{} % Empty center footer
\fancyfoot[R]{\thepage} % Page numbering for right footer
\renewcommand{\headrulewidth}{0pt} % Remove header underlines
\renewcommand{\footrulewidth}{0pt} % Remove footer underlines
\setlength{\headheight}{13.6pt} % Customize the height of the header

\numberwithin{equation}{section} % Number equations within sections (i.e. 1.1, 1.2, 2.1, 2.2 instead of 1, 2, 3, 4)
\numberwithin{figure}{section} % Number figures within sections (i.e. 1.1, 1.2, 2.1, 2.2 instead of 1, 2, 3, 4)
\numberwithin{table}{section} % Number tables within sections (i.e. 1.1, 1.2, 2.1, 2.2 instead of 1, 2, 3, 4)

\setlength\parindent{0pt} % Removes all indentation from paragraphs - comment this line for an assignment with lots of text

%----------------------------------------------------------------------------------------
%	TITLE SECTION
%----------------------------------------------------------------------------------------

\newcommand{\horrule}[1]{\rule{\linewidth}{#1}} % Create horizontal rule command with 1 argument of height

\title{	
\normalfont \normalsize 
\textsc{Politechnika Warszawska} \\ [25pt] % Your university, school and/or department name(s)
\huge HexAncientEmpires \\ % The assignment title
}

\author{Tomasz Zieliński,  Maciej Bielecki}

\date{\normalsize\today} 

\begin{document}
\maketitle 
%------------------------------------------------

\section{Wstęp}
Nasz projekt to odtworzenie częci aspektów gry Ancient Empires II wraz z naszymi modyfikacjami. Najpoważniejszą zmianą było zastąpienie planszy klasycznej przez mape podzieloną na szeciokąty. Ograniczylimy trochę zawiłe szczegółyrozgrywki, by gra była bardziej przystepna dla użytkownika. Gra daziała pod systamem Android i wykorzystuje komunikację Bluetooth. 
%------------------------------------------------

\section{Opis Gry}
HexAncientEmpires jest strategiczną grą turową. Celem gry jest wyeliminowanie wszystkich jednostek przeciwniaka za urzywając umiejentnie swoich. 
\subsection{Opis Tury}
Rozgrywkę rozpoczyna gracz pierwszy. Do dyspozycji ma jednostki w kolorze odpowiadającym numerowi gracza. Są one rozmieszczone na mapie.
 W swojej turze gracz jednokrotnie poruszyć i jednokrotnie zaatakować każdą ze swoich jednostek. Gdy gracz uzna za słuszne, by zakończyć turę powinien kliknąć przycisk "finish". Ani poruszenie i zaatakownie jednostką w czasie tury nie jest konieczne, a nawet po wykonaniu tych akci warto jest przyjrzeć się polu bitwy, więc 
\subsection{Akcje Jednostki}
Jednostki poruszają się po mapie w nietrywialny sposób, ponieważ mogą znajdować się tylko na siatce, a koszt wejcia na konkretny typ podłoża jest rózny. Żeby poruszyć jednostką należy umiecić na niej kursor, a następnie wybrać pozycje z tych oznaczonych przez kolor niebieski. Tylko takie miejsca są w zasięgu poruszania danej jednostki. Jednostka następnie wykona płynny ruch reprezentowany animacją i znajdze się w wyznaczonym celu. Przy okazji zostanie wywietlony komunikat dotyczący ruchu jednostki. Od tego momentu nie może poruszyc się aż do końca tury. Jednostka jest zdolan do atku jeżeli w jej zasięgu znajduje się jednostka innego gracza. Fakt ten jest oznajmianyprzez podwietlenie pola z jednostką przeciwnika na czerwono, gdy umiecimy kursor na naszej jednosce. Gdy klikniemy na jednostkę przeciwnika i wykonanie ataku jest możliwe, nasza jednostka zada obrażenia jednostce przeciwnika. Jeżeli jednostka przeciwnika ma odpowiedni zasięg by odpowiedzieć na nasz atak naszej jednosce zadane zostaną odpowiednie obrażenia. Wartoć ataku, zasięg poruszanie i zasięg ataku jednostek zależy od ich typu. Jednostka będąc atakowana otrzymuje obrażenia zmniejszone o obrone pola, na którym stoi.
%------------------------------------------------

\section{Architektura}
Aplikacja podzielona jest na trzy komponenty obsługiwane przez system Android:

\begin{itemize}

  \item \emph{ConnectActivity} - początkowy stan aplikacji. Wyświetla listę
    urządzeń Bluetooth i pozwala użytkownikowy wybrać, z którym z nich się
    połączyć.

  \item \emph{ConnectionService} - serwis pracujący w tle zajmujący się obsługą
    połączenia Bluetooth. Jego istnienie jest niezbędne, ponieważ system
    Android nie pozwala na przekazanie otwartego gniazda sieciowego z jednego
    \textit{Activity} do innego. Zamiast tego zarówno \textit{ConnectActivity}
    i \textit{GameActivity} łączą się z serwisem i poprzez niego wysyłają i
    odbierają komunikaty.

  \item \emph{GameActivity} - ekran obsługujący właściwą grę.

\end{itemize}



%------------------------------------------------


%------------------------------------------------
\end{document}
